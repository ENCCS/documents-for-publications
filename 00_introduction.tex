\section{Introduction}


\par
High Performance Computing (HPC) refers to the use of advanced computing technologies to solve complex problems and perform large-scale simulations or calculations at speeds and scales that are beyond the capabilities of typical desktop or server computers~\cite{hpc}.
HPC integrates systems administration (including network and security knowledge) and parallel programming into a multidisciplinary field that combines digital electronics, computer architecture, system software, programming languages, algorithms and computational techniques~\cite{hpc}.


\par
As one of the most important hardware processors, the Central Processing Unit (CPU) is known for its general-purpose and substantial processing power to execute parallel and computationally intensive tasks in HPC.
Besides CPU, the graphics processing unit (GPU) is a specialized hardware accelerator initially designed to accelerate tasks related to graphics rendering and parallel computation, and have recently been evolved to be powerful, general-purpose, and fully programmable processors, ideally suited to tackle massively parallel computing problems due to their many-core architectures~\cite{gpu_wiki}.


\par
GPU Programming involves writing code to harness the computational power of GPUs for specific tasks, including parallel computing, data processing, and scientific simulations.
However, mastering GPU programming techniques requires intensive expertise in parallel programming, memory management, extensive practical experience, and a commitment to continuous learning due to the continuous development of GPU architectures.
Therefore in this Best Practice Guide (BPG), we provide a systematic and comprehensive information for GPU programming allowing developers to leverage the massively parallel architecture of GPUs to accelerate applications in various domains.
It should be noted that this BPG extends the previously developed series of BPGs~\cite{prace-bpg} (these older guides are still relevant as they provide valuable background information about modern accelerators~\cite{modern-accelerators}, modern processors~\cite{modern-processors}, general purpose GPUs~\cite{gpgpu}, $etc.$)., but mainly covers GPU programming issues focusing on a wide range of programming languages, software environments, and APIs, such as~\textbf{directive-based programming models} (Section~\ref{sec:directive-based-programming-models}),~\textbf{non-portable kernel-based programming models} (Section~\ref{sec:non-portable-kernel-based-programming-models}), and~\textbf{portable kernel-based programming models} (Section~\ref{sec:portable-kernel-based-programming-models}).
In addition, representative high-level frameworks and libraries developed for GPU programming, the general procedures for porting code from CPU to GPU and between different GPU framerowks (Section~\ref{sec:porting_code}) are also discussed in this BPG.
At the last section~\ref{sec:stencil_example}, we provide multiple code samples for the two-dimensional heat diffusion problem using the Stencil technique. 
